
\documentclass[11pt,a4paper]{article}

% Packages
\usepackage[utf8]{inputenc}
\usepackage[T1]{fontenc}
\usepackage{amsmath}
\usepackage{booktabs}
\usepackage{graphicx}
\usepackage{siunitx}
\usepackage{geometry}
\usepackage{hyperref}
\usepackage{caption}
\usepackage{subcaption}

% Page setup
\geometry{margin=2.5cm}

% SI units setup
\sisetup{
    separate-uncertainty = true,
    multi-part-units = single
}

\title{Gait Analysis Report\\
\large Experiment: DEMO001}
\author{Stride Labs\\
\small Stride Labs}
\date{2026-01-13}

\begin{document}

\maketitle

% ============================================================================
\section{Experiment Information}
% ============================================================================

\begin{table}[h]
\centering
\caption{Experiment and subject details}
\begin{tabular}{ll}
\toprule
\textbf{Parameter} & \textbf{Value} \\
\midrule
Experiment ID & DEMO001 \\
Subject ID & MOUSE_TEST \\
Species & Mus musculus \\
Strain & C57BL/6 \\
Sex & Unknown \\
Age & N/A \\
Weight & N/A \\
Apparatus & Treadmill \\
Software Version & 1.0.0 \\
Date & 2026-01-13 \\
\bottomrule
\end{tabular}
\end{table}

% ============================================================================
\section{Methods}
% ============================================================================

Locomotion was recorded using a treadmill apparatus. Video was acquired at
\SI{30}{\hertz} and analyzed using Mouse Locomotor Tracker (v1.0.0).
Position tracking was performed using motion-based detection. Spatiotemporal
gait parameters were calculated following established protocols
\cite{catwalk2023, digigait2021}.

% ============================================================================
\section{Results}
% ============================================================================

\subsection{Spatiotemporal Parameters}

\begin{table}[h]
\centering
\caption{Spatiotemporal gait parameters. Values are mean $\pm$ SD.}
\label{tab:spatiotemporal}
\begin{tabular}{lcc}
\toprule
\textbf{Parameter} & \textbf{Value} & \textbf{Unit} \\
\midrule
Velocity & $10.53 \pm 11.09$ & \si{\centi\meter\per\second} \\
Maximum Velocity & 84.01 & \si{\centi\meter\per\second} \\
Stride Length & $1.26 \pm 1.20$ & \si{\centi\meter} \\
Cadence & $3.29 \pm 2.07$ & steps/s \\
Step Cycle & $0.402 \pm 0.204$ & \si{\second} \\
Swing Time & 40\% & - \\
Stance Time & 60\% & - \\
Duty Factor & 0.60 & - \\
\bottomrule
\end{tabular}
\end{table}

\subsection{Kinematic Parameters}

\begin{table}[h]
\centering
\caption{Kinematic parameters derived from velocity profile.}
\label{tab:kinematic}
\begin{tabular}{lcc}
\toprule
\textbf{Parameter} & \textbf{Value} & \textbf{Unit} \\
\midrule
Mean Acceleration & $69.45 \pm 125.37$ & \si{\centi\meter\per\second\squared} \\
Peak Acceleration & 600.77 & \si{\centi\meter\per\second\squared} \\
Mean Jerk & $1902.92 \pm 2783.92$ & \si{\centi\meter\per\second\cubed} \\
\bottomrule
\end{tabular}
\end{table}

\subsection{Activity Classification}

The subject exhibited the following activity distribution:
\begin{itemize}
    \item Resting: 2.5\%
    \item Walking ($<$\SI{10}{\centi\meter\per\second}): 59.3\%
    \item Trotting (\SI{10}{}-\SI{25}{\centi\meter\per\second}): 34.2\%
    \item Galloping ($>$\SI{25}{\centi\meter\per\second}): 4.0\%
\end{itemize}

Total distance traveled: \SI{71.8}{\centi\meter} over \SI{6.6}{\second}.

\subsection{Gait Variability}

\begin{table}[h]
\centering
\caption{Gait variability and regularity metrics.}
\label{tab:variability}
\begin{tabular}{lc}
\toprule
\textbf{Parameter} & \textbf{Value} \\
\midrule
Regularity Index & 19.2\% \\
Stride Variability (CV) & 95.3\% \\
Velocity Stability & -5.3\% \\
Velocity CV & 105.3\% \\
\bottomrule
\end{tabular}
\end{table}

\subsection{Data Quality}

Tracking was successful for 100.0\% of frames
(199/199 frames).
A total of 15 complete strides were detected.
Overall data quality score: 56.4/100.


% ============================================================================
\section{Figures}
% ============================================================================

\begin{figure}[h]
\centering
% \includegraphics[width=0.8\textwidth]{velocity_timeseries.pdf}
\caption{Velocity profile over time. Dashed line indicates mean velocity.
Background shading indicates activity classification (gray: rest, blue: walk,
yellow: trot, red: gallop).}
\label{fig:velocity}
\end{figure}

\begin{figure}[h]
\centering
% \includegraphics[width=0.8\textwidth]{gait_summary.pdf}
\caption{Comprehensive gait analysis summary showing (A) velocity statistics,
(B) stride metrics, (C) activity distribution, (D) gait phase composition,
(E) kinematic parameters, (F) regularity metrics, (G) distance and duration,
(H) data quality, and (I) summary table.}
\label{fig:summary}
\end{figure}


% ============================================================================
\section{References}
% ============================================================================

\begin{thebibliography}{9}

\bibitem{catwalk2023}
Lakes, E.H., Allen, K.D.
\newblock CatWalk XT gait parameters: a review of reported parameters in
pre-clinical studies of multiple central nervous system and peripheral
nervous system disease models.
\newblock \textit{Front. Behav. Neurosci.} 17, 1147784 (2023).

\bibitem{digigait2021}
Mouse Specifics, Inc.
\newblock DigiGait Imaging System.
\newblock \url{https://mousespecifics.com/digigait/} (2021).

\bibitem{deeplabcut}
Mathis, A., et al.
\newblock DeepLabCut: markerless pose estimation of user-defined body parts
with deep learning.
\newblock \textit{Nat. Neurosci.} 21, 1281-1289 (2018).

\end{thebibliography}

\end{document}
